    Запишем пару уравнений Максвелла для электромагнитного поля:
    \[
        \left\{
        \begin{array}{l}
            \rotor\vec{E} = -\mu\pder{\vec{H}}{t},\\
            \rotor\vec{H} = \eps\pder{\vec{E}}{t}.
        \end{array}
        \right.
    \]
    Рассматривая распространение волны в волноводе, представим поля в виде
    \[
        \vec{E}(x,y,z) = \vec{E}(x,y) \cdot e^{i(\omega t - hz)},\quad
        \vec{H}(x,y,z) = \vec{H}(x,y) \cdot e^{i(\omega t - hz)}
    \]
    Распишем теперь векторные уравнения покоординатно:
    \[
        \left\{
            \begin{array}{l}
                \pder{E_z}{y} + ihE_y = -i\omega\mu H_x,\\
                -ihE_x - \pder{E_z}{x} = -i\omega\mu H_y,\\
                \pder{E_y}{x} - \pder{E_x}{y} = -i\omega\mu H_z,\\
                \pder{H_z}{y} + ihH_y = i\omega\eps E_x,\\
                -ihH_x - \pder{H_z}{x} = i\omega\eps E_y,\\
                \pder{H_y}{x} - \pder{H_x}{y} = i\omega\eps E_z,\\
            \end{array}
        \right.
    \]
    Отсюда можно получить выражения для поперечных компонент поля, через
    продольные:
    \[
        \left\{
        \begin{array}{l}
            \omega\mu H_x + h E_y = i\pder{E_z}{y},\\
            h H_x + \omega\eps E_y = i\pder{H_z}{x}
        \end{array}
        \right.
        \quad
        \left\{
        \begin{array}{l}
            \omega\mu H_y - h E_x = -i\pder{E_z}{x},\\
            h H_y - \omega\eps E_x = i\pder{H_z}{y}
        \end{array}
        \right.
    \]
    \[
        \left\{
        \begin{array}{l}
            E_x = -\frac{i}{g^2}\left( h\pder{E_z}{x} +
            \omega\mu\pder{H_z}{y} \right),\\
            E_y = -\frac{i}{g^2}\left( h\pder{E_z}{y} -
            \omega\mu\pder{H_z}{x} \right),\\
            H_x = \frac{i}{g^2}\left(
            \omega\eps\pder{E_z}{y} - h \pder{H_z}{x} \right),\\
            H_y = -\frac{i}{g^2}\left(
            \omega\eps\pder{E_z}{x} + h \pder{H_z}{y} \right).
        \end{array}
        \right.
    \]
    Осталось найти выражения для продольных компонент. Для этого вернёмся к
    векторным уравнениям и возьмём ротор от обеих частей первого уравнения:
    \[
        \rotor\rotor\vec{E} = -\mu \pder{}{t}\rotor\vec{H} =
        -\eps\mu\ppder{E}{t}.
    \]
    \[
        \rotor\rotor\vec{E} = \gradient\divergence\vec{E} - \Delta\vec{E},
    \]
    \[
        \Delta\vec{E} - \eps\mu\ppder{E}{t} = 0.
    \]
    Учтём теперь вид поля в волноводе, и получим
    \[
        \Delta_\perp \vec{E}(x,y) + g^2\vec{E}(x,y) = 0.
    \]
    Совершенно аналогично можно получить для магнитного поля
    \[
        \Delta_\perp \vec{H}(x,y) + g^2\vec{H}(x,y) = 0.
    \]