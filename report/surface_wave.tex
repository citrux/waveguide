По условию оптически менее плотной является первая среда. Поэтому рассмотрим
в ней распространение поверхностной волны. Для этого воспользуемся результатами
из предыдущего раздела, заменив в 1 области тригонметрические фугкции аргумента
\(x\) на гиперболические:
\[
    \left\{
    \begin{array}{l}
        E_{1z} = E_1\sh u_1 (x-a) \sin \frac{\pi n}{b}y
        e^{i(\omega t - h z)},\\
        H_{1z} = H_1\ch u_1 (x-a) \cos \frac{\pi n}{b}y
        e^{i(\omega t - h z)},\\
        E_{2z} = E_2\sin u_2 x \sin \frac{\pi n}{b}y
        e^{i(\omega t - h z)},\\
        H_{2z} = H_2\cos u_2 x \cos \frac{\pi n}{b}y
        e^{i(\omega t - h z)}.
    \end{array}
    \right.
\]
Далее, определим соотношение между поперечными волновыми числами:
\[
    \beta_1^2 = h^2 - u_1^2 + \frac{\pi n}{b},
    \beta_2^2 = h^2 + u_2^2 + \frac{\pi n}{b};
\]
\[
    u_1^2 + u_2^2 = \beta_2^2 - \beta_1^2 = \omega^2(\eps_2\mu_2 - \eps_1\mu_1)
\]
Нетрудно заметить, что замена \( u_1 \) на \( iu_1 \) и \( E_1 \) на \( -iE_1 \)
сводит эту задачу к предыдущей. Поэтому воспользуемся теми же дисперсионными
соотношениями, несколько их видоизменив \((\tg iu_1 = i\th u_1)\):
\[
    \left[
        \begin{array}{ll}
            \frac{\mu_1\th u_1 (a-c)}{u_1} + \frac{\mu_2\tg u_2 c}{u_2} = 0
            \quad & (\mu-\textit{соотношение}),\\
            -\frac{u_1\th u_1 (a-c)}{\eps_1} + \frac{u_2\tg u_2 c}{\eps_2} = 0
            \quad & (\eps-\textit{соотношение}).
        \end{array}
    \right.
\]

\subsubsection{\(\boldsymbol{\mu}\) - волны}
\[
    \left\{
    \begin{array}{l}
        E_{1x} = 0,\\
        H_{1x} = i e^{i(\omega t - h z)}\frac{h^2 + \frac{\pi^2 n^2}{b^2}}
            {u_1h}H_1 \sh u_1 x \cos \frac{\pi n}{b}y,\\
        E_{1y} = -i e^{i(\omega t - h z)}\frac{\mu_1\omega}{u_1} H_1
            \sh u_1 x \cos \frac{\pi n}{b}y,\\
        H_{1y} = -i e^{i(\omega t - h z)}\frac{\pi n}{hb}H_1
            \ch u_1 x \sin \frac{\pi n}{b}y.\\
        E_{2x} = 0,\\
        H_{2x} = i e^{i(\omega t - h z)}\frac{h^2 + \frac{\pi^2 n^2}{b^2}}
            {u_2h}H_2 \sin u_2 x \cos \frac{\pi n}{b}y,\\
        E_{2y} = -i e^{i(\omega t - h z)}\frac{\mu_2\omega}{u_2} H_2
            \sin u_2 x \cos \frac{\pi n}{b}y,\\
        H_{2y} = -i e^{i(\omega t - h z)}\frac{\pi n}{hb}H_2
            \cos u_2 x \sin \frac{\pi n}{b}y.\\
    \end{array}
    \right.
\]

\subsubsection{\(\boldsymbol{\eps}\) - волны}
\[
    \left\{
    \begin{array}{l}
        E_{1x} = i e^{i(\omega t - h z)}\frac{h^2 + \frac{\pi^2 n^2}{b^2}}
            {hu_1}E_1\ch u_1 (x-a) \sin \frac{\pi n}{b}y,\\
        H_{1x} = 0,\\
        E_{1y} = i e^{i(\omega t - h z)}\frac{\pi n}{hb}
            E_1\sh u_1 (x-a) \cos \frac{\pi n}{b}y,\\
        H_{1y} = i e^{i(\omega t - h z)}\frac{\eps_1\omega}{u_1}E_1
            \ch u_1 (x-a) \sin \frac{\pi n}{b}y,\\

        E_{2x} = -i e^{i(\omega t - h z)}\frac{h^2 + \frac{\pi^2 n^2}{b^2}}
            {hu_2}E_2\cos u_2 x \sin \frac{\pi n}{b}y,\\
        H_{2x} = 0,\\
        E_{2y} = i e^{i(\omega t - h z)}\frac{\pi n}{hb}
            E_2\sin u_2 x \cos \frac{\pi n}{b}y,\\
        H_{2y} = -i e^{i(\omega t - h z)} \frac{\eps_2\omega}{u_2}
            E_2\cos u_2 x \sin \frac{\pi n}{b}y.\\
    \end{array}
    \right.
\]
