\documentclass[a4paper]{article}
\usepackage[utf8]{inputenc}
\usepackage[russian]{babel}
\usepackage[margin=2cm]{geometry}
\usepackage[vectors,derivative]{hedmaths}
\renewcommand{\arraystretch}{2}
\renewcommand{\frac}{\dfrac}
\begin{document}
    Запишем пару уравнений Максвелла для электромагнитного поля:
    \[
        \left\{
        \begin{array}{l}
            \rotor\vec{E} = -\mu\pder{\vec{H}}{t},\\
            \rotor\vec{H} = \eps\pder{\vec{E}}{t}.
        \end{array}
        \right.
    \]
    Рассматривая распространение волны в волноводе, представим поля в виде
    \[
        \vec{E}(x,y,z) = \vec{E}(x,y) \cdot e^{i(\omega t - hz)},\quad
        \vec{H}(x,y,z) = \vec{H}(x,y) \cdot e^{i(\omega t - hz)}
    \]
    Распишем теперь векторные уравнения покоординатно:
    \[
        \left\{
            \begin{array}{l}
                \pder{E_z}{y} + ihE_y = -i\omega\mu H_x,\\
                -ihE_x - \pder{E_z}{x} = -i\omega\mu H_y,\\
                \pder{E_y}{x} - \pder{E_x}{y} = -i\omega\mu H_z,\\
                \pder{H_z}{y} + ihH_y = i\omega\eps E_x,\\
                -ihH_x - \pder{H_z}{x} = i\omega\eps E_y,\\
                \pder{H_y}{x} - \pder{H_x}{y} = i\omega\eps E_z,\\
            \end{array}
        \right.
    \]
    Отсюда можно получить выражения для поперечных компонент поля, через
    продольные:
    \[
        \left\{
        \begin{array}{l}
            \omega\mu H_x + h E_y = i\pder{E_z}{y},\\
            h H_x + \omega\eps E_y = i\pder{H_z}{x}
        \end{array}
        \right.
        \quad
        \left\{
        \begin{array}{l}
            \omega\mu H_y - h E_x = -i\pder{E_z}{x},\\
            h H_y - \omega\eps E_x = i\pder{H_z}{y}
        \end{array}
        \right.
    \]
    \[
        \left\{
        \begin{array}{l}
            E_x = -\frac{i}{g^2}\left( h\pder{E_z}{x} +
            \omega\mu\pder{H_z}{y} \right),\\
            E_y = -\frac{i}{g^2}\left( h\pder{E_z}{y} -
            \omega\mu\pder{H_z}{x} \right),\\
            H_x = \frac{i}{g^2}\left(
            \omega\eps\pder{E_z}{y} - h \pder{H_z}{x} \right),\\
            H_y = -\frac{i}{g^2}\left(
            \omega\eps\pder{E_z}{x} + h \pder{H_z}{y} \right).
        \end{array}
        \right.
    \]
    Осталось найти выражения для продольных компонент. Для этого вернёмся к
    векторным уравнениям и возьмём ротор от обеих частей первого уравнения:
    \[
        \rotor\rotor\vec{E} = -\mu \pder{}{t}\rotor\vec{H} =
        -\eps\mu\ppder{E}{t}.
    \]
    \[
        \rotor\rotor\vec{E} = \gradient\divergence\vec{E} - \Delta\vec{E},
    \]
    \[
        \Delta\vec{E} - \eps\mu\ppder{E}{t} = 0.
    \]
    Учтём теперь вид поля в волноводе, и получим
    \[
        \Delta_\perp \vec{E}(x,y) + g^2\vec{E}(x,y) = 0.
    \]
    Совершенно аналогично можно получить для магнитного поля
    \[
        \Delta_\perp \vec{H}(x,y) + g^2\vec{H}(x,y) = 0.
    \]
    А теперь к задаче.

    Рассмотрим быстрые волны, распространяющиеся в двух частях волновода.
    Запишем уравнения:
    \[
        \left\{
        \begin{array}{l}
            \Delta_\perp E_{1z} + g_1^2 E_{1z} = 0,\\
            \Delta_\perp H_{1z} + g_1^2 H_{1z} = 0,\\
            \Delta_\perp E_{2z} + g_2^2 E_{2z} = 0,\\
            \Delta_\perp H_{2z} + g_2^2 H_{2z} = 0
        \end{array}
        \right.
    \]
    Общее решение имеет вид:
    \[
        \left\{
        \begin{array}{l}
            E_{1z} =
            (A^e_1\cos u^e_1 x + B^e_1\sin u^e_1 x)
            (C^e_1\cos v^e_1 y + D^e_1\sin v^e_1 y)
            e^{i(\omega t - h_1 z)}, \\
            H_{1z} =
            (A^m_1\cos u^m_1 x + B^m_1\sin u^m_1 x)
            (C^m_1\cos v^m_1 y + D^m_1\sin v^m_1 y)
            e^{i(\omega t - h_1 z)}, \\
            E_{2z} =
            (A^e_2\cos u^e_2 x + B^e_2\sin u^e_2 x)
            (C^e_2\cos v^e_2 y + D^e_2\sin v^e_2 y)
            e^{i(\omega t - h_2 z)}, \\
            H_{2z} =
            (A^m_2\cos u^m_2 x + B^m_2\sin u^m_2 x)
            (C^m_2\cos v^m_2 y + D^m_2\sin v^m_2 y)
            e^{i(\omega t - h_2 z)}.
        \end{array}
        \right.
    \]
Теперь воспользуемся граничными условиями и несколько упростим задачу.
На контуре волновода касательная составляющая электрического поля равна нулю,
поэтому
    \[
        E_{1z} = E_1\sin u^e_1 (x-a) \sin \frac{\pi n_1}{b}y
        e^{i(\omega t - h_1 z)},\quad
        E_{2z} = E_2\sin u^e_2 x \sin \frac{\pi n_2}{b}y
        e^{i(\omega t - h_2 z)}.
    \]
Также, учтём, что на границе раздела касательная компонента поля \( \vec{E} \)
непрерывна:
    \[
        E_1\sin u^e_1 (c-a) \sin \frac{\pi n_1}{b}y
        e^{i(\omega t - h_1 z)} =
        E_2\sin u^e_2 c \sin \frac{\pi n_2}{b}y
        e^{i(\omega t - h_2 z)}.
    \]
Равенство должно выполняться при любых \( y \) и \( z \), поэтому
\[
    \boxed{h_1 = h_2 = h,\ n_1 = n_2 = n.}
\]
Для определения вида \( H_{1z} \) найдём \( E_{1x} \):
\begin{align*}
    & E_{1x} =
    -\frac{i}{g_1^2}\left( h\pder{E_{1z}}{x} + \omega\mu\pder{H_{1z}}{y} \right)
    =\\
    & = -\frac{ie^{i(\omega t - h z)}}{g_1^2}
    \left( h E_1 u^e_1 \cos u^e_1(x-a)\sin\frac{\pi n}{b}y +
    \omega\mu(A^m_1\cos u^m_1 x + B^m_1\sin u^m_1 x)
    (-C^m_1 v^m_1 \sin v^m_1 y + D^m_1 v^m_1 \cos v^m_1 y)\right).
\end{align*}
Так как \( \left.E_{1x}\right|_{y=0,b} = 0 \), то \( D^m_1 = 0 \),
\( v^m_1 = \frac{\pi m_1}{b} y \). Аналогичный результат получается и в области
2:
\[
    \left\{
    \begin{array}{l}
        H_{1z} =
        (A^m_1\cos u^m_1 x + B^m_1\sin u^m_1 x)\cos\frac{\pi m_1}{b}y
        e^{i(\omega t - h z)}, \\
        H_{2z} =
        (A^m_2\cos u^m_2 x + B^m_2\sin u^m_2 x)\cos\frac{\pi m_2}{b}y
        e^{i(\omega t - h z)}.
    \end{array}
    \right.
\]

Теперь воспользуемся компонентами \( E_{1y} \) и \( E_{2y} \):
\begin{align*}
    & E_{1y} = -\frac{i}{g_1^2}\left( h\pder{E_{1z}}{y} -
    \omega\mu\pder{H_{1z}}{x} \right) = \\
    & = -\frac{ie^{i(\omega t - hz)}}{g_1^2}\left( h E_1 \frac{\pi n}{b}
    \sin u^e_1(x-a)\cos\frac{\pi n}{b}y - \omega\mu
    u^m_1 (-A^m_1 \sin u^m_1 x + B^m_1\cos u^m_1 x)\cos\frac{\pi m_1}{b}y
   \right)
\end{align*}
В силу условия \( \left.E_{1y}\right|_{x=a} = 0 \)
\( -A^m_1 \sin u^m_1 x + B^m_1\cos u^m_1 x = H_1\sin u^m_1(x-a) \).
Аналогично, во второй области имеем \(B^m_2=0, A^m_2 = H_2\).
В силу непрерывности на границе раздела \( E_y \), имеем \( m_1 = m_2 = n \).

Выпишем полученные выражения для компонент полей:
\[
    \left\{
    \begin{array}{l}
        E_{1z} = E_1\sin u^e_1 (x-a) \sin \frac{\pi n}{b}y
        e^{i(\omega t - h z)},\\
        H_{1z} = H_1\cos u^m_1 (x-a) \cos \frac{\pi n}{b}y
        e^{i(\omega t - h z)},\\
        E_{2z} = E_2\sin u^e_2 x \sin \frac{\pi n}{b}y
        e^{i(\omega t - h z)},\\
        H_{2z} = H_2\cos u^m_2 x \cos \frac{\pi n}{b}y
        e^{i(\omega t - h z)}.
    \end{array}
    \right.
\]

Отсюда,
\[
    u^e_i = u^m_i = u_i = \sqrt{g_i^2 - \frac{\pi^2n^2}{b^2}}.
\]
\[
    \left\{
    \begin{array}{l}
        E_{1z} = E_1\sin u_1 (x-a) \sin \frac{\pi n}{b}y
        e^{i(\omega t - h z)},\\
        H_{1z} = H_1\cos u_1 (x-a) \cos \frac{\pi n}{b}y
        e^{i(\omega t - h z)},\\
        E_{2z} = E_2\sin u_2 x \sin \frac{\pi n}{b}y
        e^{i(\omega t - h z)},\\
        H_{2z} = H_2\cos u_2 x \cos \frac{\pi n}{b}y
        e^{i(\omega t - h z)}.
    \end{array}
    \right.
\]
Определим также поперечные компоненты:
\[
    \left\{
    \begin{array}{l}
        E_{1x} = -\frac{i e^{i(\omega t - h z)}}{g_1^2}
        \left(h u_1 E_1 - \omega\mu_1\frac{\pi n}{b}H_1\right)
        \cos u_1 (x-a) \sin \frac{\pi n}{b}y,\\
        H_{1x} = \frac{i e^{i(\omega t - h z)}}{g_1^2}
        \left(\omega\eps_1\frac{\pi n}{b}E_1+h u_1 H_1\right)
        \sin u_1 (x-a) \cos \frac{\pi n}{b}y,\\
        E_{1y} = -\frac{i e^{i(\omega t - h z)}}{g_1^2}
        \left(h \frac{\pi n}{b} E_1 + \omega\mu_1u_1H_1\right)
        \sin u_1 (x-a) \cos \frac{\pi n}{b}y,\\
        H_{1y} = -\frac{i e^{i(\omega t - h z)}}{g_1^2}
        \left(\omega\eps_1u_1E_1-h \frac{\pi n}{b} H_1\right)
        \cos u_1 (x-a) \sin \frac{\pi n}{b}y,\\
        E_{2x} = -\frac{i e^{i(\omega t - h z)}}{g_2^2}
        \left(h u_2 E_2 - \omega\mu_2\frac{\pi n}{b}H_2\right)
        \cos u_2 x \sin \frac{\pi n}{b}y,\\
        H_{2x} = \frac{i e^{i(\omega t - h z)}}{g_2^2}
        \left(\omega\eps_2\frac{\pi n}{b}E_2+h u_2 H_2\right)
        \sin u_2 x \cos \frac{\pi n}{b}y,\\
        E_{2y} = -\frac{i e^{i(\omega t - h z)}}{g_2^2}
        \left(h \frac{\pi n}{b} E_2 + \omega\mu_2u_2H_2\right)
        \sin u_2 x \cos \frac{\pi n}{b}y,\\
        H_{2y} = -\frac{i e^{i(\omega t - h z)}}{g_2^2}
        \left(\omega\eps_2u_2E_2-h \frac{\pi n}{b} H_2\right)
        \cos u_2 x \sin \frac{\pi n}{b}y.
    \end{array}
    \right.
\]

Запишем теперь условия непрерывности компонент при \( x = c \):
\[
\begin{array}{cl}
    E_{y1} = E_{y2}: &
    \frac{1}{g_1^2} \left(h \frac{\pi n}{b} E_1 + \omega\mu_1u_1H_1\right)
    \sin u_1 (c-a) =  \frac{1}{g_2^2}
    \left(h \frac{\pi n}{b} E_2 + \omega\mu_2u_2H_2\right)
    \sin u_2 c ;\\
    E_{z1} = E_{z2}: & E_1\sin u_1 (c-a) = E_2\sin u_2 c; \\
    H_{y1} = H_{y2}: &
    \frac{1}{g_1^2}\left(\omega\eps_1u_1E_1-h \frac{\pi n}{b} H_1\right)
    \cos u_1 (c-a) = \frac{1}{g_2^2}
    \left(\omega\eps_2u_2E_2-h \frac{\pi n}{b} H_2\right)\cos u_2 c ;\\
    H_{z1} = H_{z2}: & H_1\cos u_1 (c-a) = H_2\cos u_2 c.\\
\end{array}
\]
Выразим амплитуды во второй области через амплитуды в первой из четных уравнений
системы и подставим в нечётные:
\[
\begin{array}{l}
    \frac{1}{g_1^2} \left(h \frac{\pi n}{b} E_1 + \omega\mu_1u_1H_1\right)
    \sin u_1 (c-a) =  \frac{1}{g_2^2}
    \left(h \frac{\pi n}{b} E_1 \sin u_1(c-a) +
    \omega\mu_2u_2H_1\cos u_1 (c-a) \tg u_2 c \right) ;\\
    \frac{1}{g_1^2}\left(\omega\eps_1u_1E_1-h \frac{\pi n}{b} H_1\right)
    \cos u_1 (c-a) = \frac{1}{g_2^2}
    \left(\omega\eps_2u_2E_1\sin u_1(c-a)\ctg u_2 c-
    h \frac{\pi n}{b} H_1\cos u_1 (c-a)\right) ;\\
\end{array}
\]
перегруппируем
\[
\begin{array}{l}
    E_1\left(\frac{1}{g_1^2} h \frac{\pi n}{b}\sin u_1 (c-a) -
    \frac{1}{g_2^2}h \frac{\pi n}{b}\sin u_1(c-a)\right) =
    H_1\left(\frac{1}{g_2^2}\omega\mu_2u_2\cos u_1 (c-a) \tg u_2 c-
    \frac{1}{g_1^2}\omega\mu_1u_1\sin u_1 (c-a)\right);\\
    E_1\left(\frac{1}{g_1^2}\omega\eps_1u_1\cos u_1 (c-a)-
    \frac{1}{g_2^2}\omega\eps_2u_2\sin u_1(c-a)\ctg u_2 c\right)=
    H_1\left(\frac{1}{g_1^2}h \frac{\pi n}{b} \cos u_1 (c-a)-
    \frac{1}{g_2^2}h \frac{\pi n}{b} \cos u_1 (c-a)\right)
\end{array}
\]
Отсюда
\begin{align*}
    &\left(\frac{1}{g_1^2} h \frac{\pi n}{b}\sin u_1 (c-a) -
    \frac{1}{g_2^2}h \frac{\pi n}{b}\sin u_1(c-a)\right)\left(\frac{1}{g_1^2}h \frac{\pi n}{b} \cos u_1 (c-a)-
    \frac{1}{g_2^2}h \frac{\pi n}{b} \cos u_1 (c-a)\right) =\\
    & \left(\frac{1}{g_2^2}\omega\mu_2u_2\cos u_1 (c-a) \tg u_2 c-
    \frac{1}{g_1^2}\omega\mu_1u_1\sin u_1 (c-a)\right)
    \left(\frac{1}{g_1^2}\omega\eps_1u_1\cos u_1 (c-a)-
    \frac{1}{g_2^2}\omega\eps_2u_2\sin u_1(c-a)\ctg u_2 c\right).
\end{align*}
\[
    \left( \frac{1}{g_1^2} - \frac{1}{g_2^2} \right)^2\frac{h^2\pi^2n^2}{b^2}=
    \left(\frac{1}{g_2^2}\omega\mu_2u_2\tg u_2 c-
    \frac{1}{g_1^2}\omega\mu_1u_1\tg u_1 (c-a)\right)
    \left(\frac{1}{g_1^2}\omega\eps_1u_1\ctg u_1 (c-a)-
    \frac{1}{g_2^2}\omega\eps_2u_2\ctg u_2 c\right),
\]
\[
    \left( \frac{1}{g_1^2} - \frac{1}{g_2^2} \right)^2\frac{h^2\pi^2n^2}{b^2}=
    \frac{\omega^2u_1u_2}{g_1^2g_2^2}\left(\eps_1\mu_2\frac{\tg u_2 c}{\tg u_1
    (c-a)} + \eps_2\mu_1\frac{\tg u_1 (c-a)}{\tg u_2 c}\right) -\frac{1}{g_1^4}
    \omega^2u_1^2\eps_1\mu_1 - \frac{1}{g_2^4}\omega^2u_2^2\eps_2\mu_2.
\]
Введём обозначение
\[
    A = \frac{\tg u_2 c}{\tg u_1 (c-a)},
\]
с учётом которого уравнение принимает вид
\[
    \frac{2}{g_1^2 g_2^2}\frac{h^2\pi^2n^2}{b^2} =
    -\frac{\omega^2u_1u_2}{g_1^2g_2^2}\left(\eps_1\mu_2 A +
    \eps_2\mu_1\frac{1}{A}\right) +
    \frac{1}{g_1^4}
    \left(\omega^2u_1^2\eps_1\mu_1 + \frac{h^2\pi^2n^2}{b^2}\right) +
    \frac{1}{g_2^4}
    \left(\omega^2u_2^2\eps_2\mu_2 + \frac{h^2\pi^2n^2}{b^2}\right)
\]
Учтём, что
\[
    \omega^2u_i^2\eps_i\mu_i + \frac{h^2\pi^2n^2}{b^2} = \beta_i^2u_i^2 +
    (\beta_i^2 - g_i^2)\frac{\pi^2n^2}{b^2} =
    \beta_i^2\left( u_i^2 + \frac{\pi^2n^2}{b^2} \right) -
    g_i^2\frac{\pi^2n^2}{b^2} =
    g_i^2\left( \beta_i^2 - \frac{\pi^2n^2}{b^2}\right).
\]
Подставив, получим
\[
    \frac{2}{g_1^2 g_2^2}\frac{h^2\pi^2n^2}{b^2} =
    -\frac{\omega^2u_1u_2}{g_1^2g_2^2}\left(\eps_1\mu_2 A +
    \eps_2\mu_1\frac{1}{A}\right) +
    \frac{1}{g_1^2}
    \left(\beta_1^2 - \frac{\pi^2n^2}{b^2}\right) +
    \frac{1}{g_2^2}
    \left(\beta_2^2 - \frac{\pi^2n^2}{b^2}\right)
\]
Домножив на \( g_1^2g_2^2 \), получим:
\[
    2\frac{h^2\pi^2n^2}{b^2} = g_2^2
    \left(\beta_1^2 - \frac{\pi^2n^2}{b^2}\right) +
    g_1^2\left(\beta_2^2 - \frac{\pi^2n^2}{b^2}\right)-
    \omega^2u_1u_2\left(\eps_1\mu_2 A +
    \eps_2\mu_1\frac{1}{A}\right),
\]
Теперь как проще упростить?
\[
    2\frac{h^2\pi^2n^2}{b^2} =
    \beta_1^2 g_2^2 + \beta_2^2 g_1^2 -
    (g_1^2+g_2^2)\frac{\pi^2n^2}{b^2} -
    \omega^2u_1u_2\left(\eps_1\mu_2 A +
    \eps_2\mu_1\frac{1}{A}\right),
\]
\[
    2\frac{h^2\pi^2n^2}{b^2} =
    \beta_1^2 g_2^2 + \beta_2^2 g_1^2 -
    (\beta_1^2+\beta_2^2)\frac{\pi^2n^2}{b^2} +
    2\frac{h^2\pi^2n^2}{b^2} -
    \omega^2u_1u_2\left(\eps_1\mu_2 A +
    \eps_2\mu_1\frac{1}{A}\right),
\]
\[
    \beta_1^2u_2^2 + \beta_2^2u_1^2=
    \omega^2u_1u_2\left(\eps_1\mu_2 A +
    \eps_2\mu_1\frac{1}{A}\right),
\]
Подставляем выражение для \( \beta \) и получаем:
\[
    \eps_1\mu_1\frac{u_2}{u_1} + \eps_2\mu_2\frac{u_1}{u_2} =
    \eps_1\mu_2 A + \eps_2\mu_1\frac{1}{A}
\]
\[
    A^2 - \left(
    \frac{\mu_1}{\mu_2}\frac{u_2}{u_1} + \frac{\eps_2}{\eps_1}\frac{u_1}{u_2}
    \right)A + \frac{\eps_2}{\eps_1}\cdot\frac{\mu_1}{\mu_2} = 0.
\]
По теореме Виета
\[
    A_1 = \frac{\mu_1}{\mu_2}\frac{u_2}{u_1},\quad
    A_2 = \frac{\eps_2}{\eps_1}\frac{u_1}{u_2}.
\]
Таким образом, система имеет решение при выполнении какого-либо из условий
\[
    \frac{\tg u_2 c}{\tg u_1 (c-a)} =
    \frac{\mu_1}{\mu_2}\frac{u_2}{u_1},\quad
    \frac{\tg u_2 c}{\tg u_1 (c-a)} =
    \frac{\eps_2}{\eps_1}\frac{u_1}{u_2}.
\]
Перепишем их в виде
\[
    \left[
        \begin{array}{l}
            \frac{\mu_1\tg u_1 (a-c)}{u_1} + \frac{\mu_2\tg u_2 c}{u_2} = 0,\\
            \frac{u_1\tg u_1 (a-c)}{\eps_1} + \frac{u_2\tg u_2 c}{\eps_2} = 0.
        \end{array}
    \right.
\]

Теперь выразим все компоненты через \( E_1 \):
В случае
\[
    \frac{\mu_1\tg u_1 (a-c)}{u_1} + \frac{\mu_2\tg u_2 c}{u_2} = 0,
\]
имеем
\[
    E_1\left(\frac{1}{g_1^2} h \frac{\pi n}{b}-
    \frac{1}{g_2^2}h \frac{\pi n}{b}\right) =
    H_1\left(\frac{1}{g_2^2}\omega\mu_2u_2\frac{\tg u_2 c}{\tg u_1 (c-a)}-
    \frac{1}{g_1^2}\omega\mu_1u_1\right),
\]
\[
    E_1\frac{h\pi n}{b}\left(\frac{1}{g_1^2}-\frac{1}{g_2^2}\right)=
    H_1\left(\frac{1}{g_2^2}\omega\mu_2u_2\frac{\mu_1u_2}{\mu_2u_1}-
    \frac{1}{g_1^2}\omega\mu_1u_1\right),
\]
\[
    H_1 = E_1\frac{u_1h\pi n}{\mu_1\omega b}
    \frac{g_2^2-g_1^2}{g_1^2u_2^2 - g_2^2u_1^2} =
    E_1\frac{u_1hb}{\mu_1\omega\pi n},
\]
\[
    E_2 = E_1\frac{\sin u_1(c-a)}{\sin u_2 c},
\]
\[
    H_2 = H_1 \frac{\cos u_1(c-a)}{\cos u_2 c} =
    E_1\frac{u_1hb}{\mu_1\omega\pi n}\frac{\tg u_2c}{\tg u_1(c-a)}
    \frac{\sin u_1(c-a)}{\sin u_2 c} = E_1\frac{u_2hb}{\mu_2\omega\pi n}
    \frac{\sin u_1(c-a)}{\sin u_2 c}.
\]
А в случае
\[
    \frac{u_1\tg u_1 (a-c)}{\eps_1} + \frac{u_2\tg u_2 c}{\eps_2} = 0,
\]
имеем
\[
    E_1\frac{h\pi n}{b}\left(\frac{1}{g_1^2}-\frac{1}{g_2^2}\right)=
    H_1\left(\frac{1}{g_2^2}\omega\mu_2u_2\frac{u_1\eps_2}{u_2\eps_1}-
    \frac{1}{g_1^2}\omega\mu_1u_1\right),
\]
\[
    H_1 = E_1\frac{\eps_1h\pi n}{u_1\omega b}
    \frac{g_2^2-g_1^2}{g_1^2\eps_2\mu_2 - g_2^2\eps_1\mu_1} =
    -E_1\frac{\eps_1\omega\pi n}{u_1hb},
\]
\[
    E_2 = E_1\frac{\sin u_1(c-a)}{\sin u_2 c},
\]
\[
    H_2 = H_1 \frac{\cos u_1(c-a)}{\cos u_2 c} =
    -E_1\frac{\eps_1\omega\pi n}{u_1hb}\frac{\tg u_2c}{\tg u_1(c-a)}
    \frac{\sin u_1(c-a)}{\sin u_2 c} = -E_1\frac{\eps_2\omega\pi n}{u_2hb}
    \frac{\sin u_1(c-a)}{\sin u_2 c}.
\]
\end{document}

